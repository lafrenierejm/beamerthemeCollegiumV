\documentclass{beamer}
\usetheme{CollegiumV}
\usecolortheme{ComingClean}
% \usecolortheme{Entrepreneur}
% \usecolortheme{ConspiciousCreep}  %% VERY garish.

\usepackage[utf8]{inputenc}
\usepackage[T1]{fontenc}
\usepackage{libertine}
\usepackage[scaled=0.92]{inconsolata}
\usepackage[libertine]{newtxmath}

% Bibliography
\usepackage[backend=biber,style=ieee]{biblatex}
\bibliography{bibliography.bib}

% dimensions
\usepackage{calc}
\usepackage{siunitx}
\usepackage{printlen}
\beamertemplategridbackground[1in]
\newlength{\colsep}
\newlength{\colwidth}
\setlength{\colsep}{0.5in}
\setlength{\colwidth}{0.33\textwidth-0.67\colsep}

\usepackage{mwe}

\author{Your Name Here}
\title{Your Poster Title}
\major{Your Major}

\begin{document}

% \setlength{\textwidth}{35in}
\begin{frame}[c,fragile]
  \begin{block}{Full width}
    This block is the full \texttt{textwidth} because it is not inside a column (in fact, it's not in a \texttt{columns} environment at all).
    The following lengths are displayed using the \texttt{printlen} package.
    You can probably remove that package for your poster.
    \begin{itemize}
    \item
      Paperwidth is \uselengthunit{in}\printlength{\paperwidth}.
    \item
      Paperheight is \uselengthunit{in}\printlength{\paperheight}.
    \item
      Textwidth is \uselengthunit{in}\printlength{\textwidth}.
    \item
      Colwidth is \uselengthunit{in}\printlength{\colwidth}.
    \end{itemize}
  \end{block}
  \begin{columns}[T,onlytextwidth]
    \begin{column}{\colwidth}
      \begin{block}{Useful Commands}
        \begin{enumerate}
        \item
          You can insert the contents of \path{\author} using \path{\insertauthor}: \insertauthor{}.
        \item\label{email}
          Email address can be formatted as \texttt{mailto} URLs with the \path{\email} macro.
          \path{\email{email@address}} becomes a clickable \email{email@address}.
        \item\label{emailcontact}
          To wrap an email address in angle braces, use \path{\emailcontact}.
          \path{\emailcontact{email@address}} becomes \emailcontact{email@address}.
        \item
          The font style of items \ref{email} and \ref{emailcontact} is provided by the \texttt{hyperref} package and can be set with \path{\urlstyle}.
          \begin{itemize}
          \item
            This is an example of an arbitrary second-level \texttt{itemize}.
          \end{itemize}
        \end{enumerate}
      \end{block}
    \end{column}

    % second column
    \begin{column}{\colwidth}
      \begin{block}{Dimensions}
        If you read the source for this poster you will see definitions of \path{\colsep} and \path{\colwidth}, where \path{\colwidth} is defined in terms of \path{\textwidth} and \path{\colsep}.
        The \texttt{calc} package is used to perform the calculation needed for \path{\colwidth}.
        The value of \path{\textwidth} used by \path{\colwidth} is a constant defined in the ColleviumV theme as 36 inches per the templates provided at \url{honors.utdallas.edu/cv/senior-thesis} as of 2018-04-17.
        The default value of \path{\colwidth} is appropriate for a three-column layout.
      \end{block}
    \end{column}

    % third column
    \begin{column}{\colwidth}
      \begin{block}{Overview}
        \begin{itemize}
        \item
          This is the template I created for my poster presentations.
        \item
          In this bullet I reference \textit{Scrum and XP from the Trenches}.~\cite{kniberg}
        \item
          In this bullet I reference a work by Kent Beck.~\cite{agile_principles}
        \end{itemize}
      \end{block}
      \begin{block}{References}
        \renewcommand*{\bibfont}{\normalfont\footnotesize}
        \setbeamertemplate{bibliography item}[text]
        \printbibliography{}
      \end{block}
    \end{column}
  \end{columns}
\end{frame}

\end{document}

%%% Local Variables:
%%% mode: xelatex
%%% TeX-master: t
%%% End:
